\documentclass[10pt]{article}
\usepackage{graphicx} % Required for including images
\usepackage[T1]{fontenc} % Use 8-bit encoding that has 256 glyphs
\usepackage[utf8]{inputenc} % Required for including letters with accents
\usepackage{amsmath,amssymb,amsthm} % For including math equations, theorems, symbols, etc
\usepackage{quotes} % for quotation marks

\title{Projektskizze}

\begin{document}

\begin{titlepage}

\raggedleft % Right align everything
	
	\vspace*{\baselineskip} % Whitespace at the top of the page
	
	\rightline{{\large Claudio Frei, Karin Birle, Leo Rudin,}} 
	\rightline{\large Michael Wigger, Nathalie Achtnich, Philippe Schneider,} 
	\rightline{\large Raphael Krebs, Silvan Nigg, Stefan Enz}
	
	\vspace*{0.167\textheight} % Whitespace before the title
	
	\textbf{\LARGE Virtual CT Board}\\[\baselineskip] % First title line
	
	\Huge Projektskizze\\[\baselineskip] % Main title line 
	
	\vfill % Whitespace 
	
	{\large ZHAW - FS 2022}
	
	\vspace*{3\baselineskip} % Whitespace at the bottom of the page


\end{titlepage}

\tableofcontents

\newpage 

\section{Ausgangslage}

Jedes Jahr beginnen fast 200 Studentinnen und Studenten ein Bachelorstudium der Informatik an der ZHAW. Zu ihren Pflichtmodulen in den ersten Jahren gehört die Veranstaltung "Computertechnik 1", die die grundlegende Funktionsweise eines Computers an der Schnittstelle zwischen Hardware und Software behandelt. Im dazugehörigen Praktikum probieren die Studenten das erworbene Wissen selbst aus und benutzen dabei als Hardware ein sogenanntes CT Board, das über einen Prozessor und einen Speicher verfügt und diverse Input-/Output-Möglichkeiten bereitstellt (siehe Abbildung XY). Damit sie die Übungen auch zuhause durchführen können, leihen die Studenten zu Beginn des Semesters ein solches CT Board von den Dozenten aus. Dieses muss von ihnen nach Hause transportiert und am Ende des Jahres wieder zurückgebracht werden.

\section{Idee}

Im 21. Jahrhundert noch Hardware an Studenten zu verleihen, mutet schon fast archaisch an.

\section{Kundennutzen}

\section{Stand der Technik / Konkurrenzanalyse}

\section{Kontextszenario (Hauptablauf)}

\section{Weitere Anforderungen}

\section{Ressourcen}

\section{Risiken}

\section{Wirtschaftlichkeit}

\end{document}