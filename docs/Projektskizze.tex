\documentclass[10pt]{article}
\usepackage{graphicx} % Required for including images
\usepackage[T1]{fontenc} % Use 8-bit encoding that has 256 glyphs
\usepackage[utf8]{inputenc} % Required for including letters with accents
\usepackage{amsmath,amssymb,amsthm} % For including math equations, theorems, symbols, etc
\usepackage{quotes} % for quotation marks

\title{Projektskizze}

\begin{document}

\begin{titlepage}

\raggedleft % Right align everything
	
	\vspace*{\baselineskip} % Whitespace at the top of the page
	
	\rightline{{\large Claudio Frei, Karin Birle, Leo Rudin,}} 
	\rightline{\large Michael Wigger, Nathalie Achtnich, Philippe Schneider,} 
	\rightline{\large Raphael Krebs, Silvan Nigg, Stefan Enz}
	
	\vspace*{0.167\textheight} % Whitespace before the title
	
	\textbf{\LARGE Virtual CT Board}\\[\baselineskip] % First title line
	
	\Huge Projektskizze\\[\baselineskip] % Main title line 
	
	\vfill % Whitespace 
	
	{\large ZHAW - FS 2022}
	
	\vspace*{3\baselineskip} % Whitespace at the bottom of the page


\end{titlepage}

\tableofcontents

\newpage 

\section{Ausgangslage}

Jedes Jahr beginnen fast 200 Studentinnen und Studenten ein Bachelorstudium der Informatik an der ZHAW. Zu ihren Pflichtmodulen in den ersten Jahren gehört die Veranstaltung "Computertechnik 1", die die grundlegende Funktionsweise eines Computers an der Schnittstelle zwischen Hardware und Software behandelt. Im dazugehörigen Praktikum probieren die Studenten das erworbene Wissen selbst aus und benutzen dabei als Hardware ein sogenanntes CT Board, das über einen Prozessor und einen Speicher verfügt und diverse Input-/Output-Möglichkeiten bereitstellt (siehe Abbildung XY). Damit sie die Übungen auch zuhause durchführen können, leihen die Studenten zu Beginn des Semesters ein solches CT Board von den Dozenten aus. Dieses muss von ihnen nach Hause transportiert und am Ende des Jahres wieder zurückgebracht werden.

\section{Idee}

Im 21. Jahrhundert noch Hardware an Studenten zu verleihen, mutet schon fast archaisch an. Insbesondere, wenn sich die Hardware leicht als Software simulieren liesse. Und genau hier setzt unsere Idee an: Wir möchten eine Web-Applikation entwickeln, die ein CT Board simuliert und von den Studenten anstelle dessen für die "Computertechnik 1"-Praktika eingesetzt werden kann. Die Studenten können die Übungen mit unserem leichtgewichtigen "Virtual CT Board" genauso gut wie mit dem echten Board lösen und brauchen dafür fortan keine Hardware mehr. Die ZHAW kann sich somit die Beschaffung und Wartung von CT Boards sparen.

\section{Kundennutzen}

Unsere Kundschaft besteht aus zwei Zielgruppen, nämlich die Informatikstudenten der ZHAW auf der einen Seite und die ZHAW als Institution auf der anderen Seite. Für beide Gruppen ergeben sich durch unser Produkt unmittelbare Vorteile:

\paragraph{Für die Studenten:}
\begin{itemize}
\item Die Studenten des Moduls "Computertechnik 1" haben eine elegante Alternative zu den physischen CT Boards und müssen jene nicht mehr nach Hause tragen.
\item Im Gegensatz zum bisherigen physischen CT Board benötigt unser "Virtual CT Board" keinen Strom und lässt sich daher überall - also beispielsweise auch unterwegs - direkt auf dem Laptop starten.
\item Die bisher für den C- und Assembler-Code eingesetzte Software Keil wird überflüssig und muss nicht mehr händisch konfiguriert werden.
\item Es ergeben sich keine Unklarheiten mehr, wie das CT Board mit dem eigenen Computer verkabelt und von diesem aus bedient werden muss.
\item Die Studenten müssen sich nicht mehr darum sorgen, dass ihr geliehenes CT Board kaputtgehen könnte.
\end{itemize}

\paragraph{Für die ZHAW:}
\begin{itemize}
\item Die Lizenz für unsere Software ist um einiges preiswerter als die Beschaffung physischer CT Boards. Die ZHAW spart also Geld.
\item Alte oder defekte Geräte müssen nicht mehr ausgetauscht oder repariert werden. Auch dies spart Geld und Zeit.
\item Die Dozenten müssen für die Praktika nicht mehr mühsam Keil-Projekte konfigurieren. Es genügt, die Assembler-Files ins "Virtual CT Board" hochzuladen.
\end{itemize}

\section{Stand der Technik / Konkurrenzanalyse}

\section{Kontextszenario (Hauptablauf)}

\section{Weitere Anforderungen}

\section{Ressourcen}

\section{Risiken}

\section{Wirtschaftlichkeit}

\end{document}